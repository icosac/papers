% This is samplepaper.tex, a sample chapter demonstrating the
% LLNCS macro package for Springer Computer Science proceedings;
% Version 2.21 of 2022/01/12
%
\documentclass[runningheads]{llncs}
%
\usepackage[T1]{fontenc}
% T1 fonts will be used to generate the final print and online PDFs,
% so please use T1 fonts in your manuscript whenever possible.
% Other font encondings may result in incorrect characters.
%
% packages for acronyms
\usepackage{glossaries}
\usepackage{acronyms}

% math packs
\usepackage{amsthm}
\usepackage{bm} % bold math
\let\yesnumberold=\yesnumber\relax
\let\yesnumber\relax
\usepackage[math]{easyeqn}
\let\yesnumber\yesnumberold

\usepackage{graphicx}
% Used for displaying a sample figure. If possible, figure files should
% be included in EPS format.
%
% If you use the hyperref package, please uncomment the following two lines
% to display URLs in blue roman font according to Springer's eBook style:
\usepackage{color}
\renewcommand\UrlFont{\color{blue}\rmfamily}

\usepackage{url}
\usepackage{makecell}
\usepackage{multirow}

%Table commands
\newcommand{\mc}[3]{\multicolumn{#1}{#2}{#3}}

% function names
\newcommand{\sinc}{\mathop{\mathrm{sinc}}}
\newcommand{\cosc}{\mathop{\mathrm{cosc}}}
\newcommand{\ham}{\mathcal{H}}
\newcommand{\dist}{\mathrm{dist}}
\newcommand{\vvv}{``}
\newcommand{\argmin}{\mathop{\mathrm{argmin}}}
\newcommand{\modpi}{\mathrm{mod2pi}}
\newcommand{\atandue}{\mathrm{atan2}}
\newcommand{\DEF}{:=}

% maths
%Math commands
\newcommand{\Sum}[0]{\displaystyle\sum}
\newcommand{\Prod}[0]{\displaystyle\prod}


%
\title{Comparing Multi-Agent Path Finding Algorithms in a Real Industrial Scenario}
%
%\titlerunning{Abbreviated paper title}
% If the paper title is too long for the running head, you can set
% an abbreviated paper title here
%
\author{Enrico Saccon\orcidID{0000-0002-2418-6618} \and
Luigi Palopoli\orcidID{0000-0001-8813-8685} \and
Marco Roveri\orcidID{0000-0001-9483-3940}}
%
\authorrunning{E. Saccon et al.}
% First names are abbreviated in the running head.
% If there are more than two authors, 'et al.' is used.
%
\institute{
  University of Trento\\
  \email{enrico.saccon@studenti.unitn.it}\\
  \email{\{luigi.palopoli,marco.roveri\}@unitn.it} \\
  \url{https://www.unitn.it}
}

%
\begin{document}
%
\maketitle              % typeset the header of the contribution
%
%
%

\begin{abstract}

  There is an increasing trend for automating warehouses and factories
  leveraging on teams of autonomous robotic agents. The orchestration
  problem for a fleet of autonomous robotic cooperating agents has
  been tackled in the literature as Multi-Agent Path Finding (MAPF).
  Several algorithms have been proposed in the literature to solve the
  MAPF problem. However, these algorithms have been only applied to
  synthetic randomly generated scenarios. The application in real
  application demands for scalability (being able to deal with
  realistic size warehouses) and efficiency (being able to quickly
  adapt to changes in the problems due e.g.  new orders or change in
  priorities).

  In this work we perform a detailed analysis of the MAPF literature,
  we selected the most effective algorithms, we implemented them and
  we carried out an experimental analysis on a real scalable warehouse
  of a large distribution company to evaluate their applicability in
  such scenarios.

  The results show that a) there is not a clear winner; b) there are
  difficult (realistic) cases out of scope of all the algorithms. For
  such difficult cases, we designed novel heuristics to be employed by
  the best algorithms to solve them efficiently.
\end{abstract}

\endinput
\begin{abstract}
The new industrial revolution is proving the critical role of robotics in
industry and also in everyday lives of people. This work focuses on the
Multi-Agent Path Finding (MAPF) considering a real warehouse located in the
center of Italy as the main scenario. \newline
First, we start with a description of the state-of-the-art algorithms
used to solve the Single-Agent Path Finding (SAPF) problem since they are used
as a building block for many algorithms solving the MAPF problem. After that we
move to a precise description of the algorithms that solve the main problem
showing their strengths and their weak spots. In particular, we focus on the
description of correct and optimal solvers. Next we propose a concise review of
some variants of the classical MAPF problem explaining why some of these
aspects are important to model our problem on the warehouse. Then, we describe
the three algorithms we have chosen and implemented to tackle the
aforementioned problem explaining the modification we had to make to them in
order to be used in our scenario. Two of these are based on the same MAPF
algorithm, but they use different internal solvers as it will be more clear
later on. Finally the third algorithm is based on constraint programming, which
differs from all the other implementations. \newline
After that, we also delineate a suite of tests taken from the warehouse map of
increasing difficulty. Next a discussion on the results and the future work
that may come with it is carried out. The results show the implemented
algorithms’ limitations, from the analysis of which we describe a particularly
difficult case. Eventually, we propose some modifications to the algorithms
that may improve the performance and the number of cases they manage to solve.
\end{abstract}


%
% ---- Bibliography ----
%
% BibTeX users should specify bibliography style 'splncs04'.
% References will then be sorted and formatted in the correct style.
%
\bibliographystyle{splncs04}
\bibliography{mybibliography}

\end{document}
